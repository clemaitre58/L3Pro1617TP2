\documentclass[12pt]{tdtp}
\usepackage{tabularx,colortbl}
\usepackage{multirow}
\usepackage{listings}
\lstset{
	language=VHDL,
basicstyle=\tiny\ttfamily}
\definecolor{light-gray}{gray}{0.96}
\definecolor{pageheading-gray}{gray}{0.2}
\definecolor{dark-gray}{gray}{0.45}
\definecolor{dark-green}{rgb}{0.245,0.121,0.0}

\newcommand{\auteur}{Cedric Lemaitre}
\newcommand{\couriel}{c.lemaitre58@gmail.com}
\newcommand{\promo}{L3 Pro Robotique}
\newcommand{\annee}{2017-2018}
\newcommand{\matiere}{Traitement M3.1}

\newcommand{\tdtp}{TP 2}
\renewcommand{\sujet}{Caractérisation d'images}


\begin{document}
\titre
Ce TP propose de réaliser la phase d'acquisition et la caractérisation des données d'un système décisionnel basé sur l'image comme information caractéristique\\
\\
\\
\\\

%%%%%%%%%%%
\Exo

Nous devons dans un premier temps créer une base d'apprentissage. Pour cela il est nécessaire de réaliser l'acquisition d'objet.

Nous considérons 3 de vos objets usuels (par exemple un smartphone, un stylo, une clé).

Prenez \textbf{individuellement} chaque objet en photo à l'aide de votre smartphone.

Pour chaque objet, prenez une dizaine de photos selon différents points de vue.

Importer ces images dans votre répertoire personnel.

Répartissez les images dans 3 répertoires spécifiques, c'est à dire un répertoire par objet.


%%%%%%%%%%%%
\Exo

Nous allons procéder à la caractérisation des objets.

Pour cela, un jeu de caractéristiques sera calculé pour chaque image.

A l'aide de la librairie \textit{scikit-image}, nous avons procédé (pour une seule image) de la manière suivante :

\begin{enumerate}
	\item ouvrir l'image à l'aide de la fonction \textit{imread}
	\item calculer la valeur moyenne et l'écart type des quantités suivantes: $C_1 = \frac{G}{R+G+B}$, $C_2 = \frac{B}{R+G+B}$  et $C_1 = \frac{R+G+B}{3*255}$ 
\end{enumerate}

%%%%%%%%%%%
\Exo 

Modifier votre script pour que le calcul des caractéristiques précédentes soit réalisé sur l'ensemble des image.

Pour cela, utiliser la librairie \textbf{os} et en particulier la fonction \textbf{walk}

Stocker dans les résultats dans 2 matrices $X$ et $Y$. 

Chaque ligne de $X$ contient les caractéristiques d'une image. Chaque ligne de $Y$ contient le label d'appartenance d'une image.
\\
\textbf{NB : il est crucial que les données dans $X$ et $Y$ respecte le même ordre}

\end{document}
